\documentclass[12pt,a4paper,oneside,draft]{article}
\title{Homework One: Problem Three}
\author{Corrina Black}
\date{April 2014}
\linespread{1.1}
\usepackage{amsmath}
\usepackage{color}

\begin{document}
\noindent\textsc{\Large\textbf{Homework One: Problem Three}}
\newline
\newline
\normalsize  
5 Points: Solve the following exercises and submit your answers in a pdf file using any formatting (+2 points for using LATEX formatting): 
\begin{enumerate}
     \item What are the approximate absolute and relative errors in approximating $\pi$ by each of the following quantities (\textbf{Absolute error} = approximate value - true value and \textbf{Relative error} = (approximate value - true value)/true value)?
	\begin{enumerate}
    	\item 3
    	\newline
    	\textcolor[rgb]{1,.5,0}{Absolute error = (3-$\pi$) = -0.1415926535897932384626433832795}
    	\newline
    	\textcolor[rgb]{1,.5,0}{Relative error = (3-$\pi$)/$pi$ = -0.04507034144862798538669741976491}
    	\item 3.14
    	\newline
    	\textcolor[rgb]{1,.5,0}{Absolute error = (3.14-$\pi$) = -0.0015926535897932384626433832795}
    	\newline
    	\textcolor[rgb]{1,.5,0}{Relative error = (3.14-$\pi$)/$pi$ = -5.0695738289729137140996602060981e-4}
     	\item 22/7
    	\newline
    	\textcolor[rgb]{1,.5,0}{Absolute error = ((22/7)-$\pi$) = 0.00126448926734961868021375957764}
    	\newline
    	\textcolor[rgb]{1,.5,0}{Relative error = ((22/7)-$\pi$)/$pi$ = 4.0249943477068197584079834151885e-4}
     \end{enumerate}
     
     \item Consider the problem of evaluating the function sin(x), in particular, the propagated data
error, i.e., the error in the function value due to a perturbation h in the argument x.
	\begin{enumerate}
    	\item Estimate the absolute error in evaluating sin(x)
    	\newline
    	\textcolor[rgb]{1,.5,0}{Absolute error = DON'T UNDERSTAND HOW TO FIGURE THIS OUT}
    	\item Estimate the relative error in evaluating sin(x)
    	\newline
    	\textcolor[rgb]{1,.5,0}{Relative error = DON'T UNDERSTAND HOW TO FIGURE THIS OUT}
     	\item Estimate the condition number for this problem.
    	\newline
    	\textcolor[rgb]{1,.5,0}{Condition number = (sin(x+h)-sin(x)/sin(x))/((x+h)-x/x)}
     	\item For what values of the argument x is this problem highly sensitive?
    	\newline
    	\textcolor[rgb]{1,.5,0}{If you look at the graph for sin(x) you find curve peaks at ..., -2$\pi$, -$\pi$, 0, $\pi$, 2$\pi$, ... and this is where it's highly sensitive as the derivative changes drastically }\ldots
     \end{enumerate}
\end{enumerate}
\end{document}